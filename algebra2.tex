\documentclass{article}
\usepackage{kotex} % for korean support
\usepackage{color}
\usepackage{mathtools} % for pmatrix command - 행렬 표시
\usepackage{hyperref} % for hyperlink
\usepackage[margin=0.5in]{geometry} % for document margin
\usepackage{polynom} % automatic polynomial calculations(divisions, more specifically)
\title{Algebra II \\대수학2}
\author{박성렬}

\begin{document}
\maketitle

\section{용어}
\begin{itemize}
\item \textbf{?mial/?항식}\\
\begin{tabular} {| l | l | l | l |}
\hline
monomial & 1항식 & $a$\\
\hline
binomial & 2항식 & $a+b$\\
\hline
trinomial & 3항식 & $a+b+c$\\
\hline
polynomial & 다항식 & $a+...$\\
\hline
\end{tabular}

\item \textbf{binomial theorem/2항식의 법칙}
$$(a+b)^{n}= \sum_{k=0}^{n} \begin{pmatrix}n \\ k \end{pmatrix} a^{n-k}b^{k}$$
$$\begin{pmatrix} n \\k \end{pmatrix}=\dfrac{n!}{k!(n-k)!} $$
\item \textbf{imaginary number/허수}:
직선으로 된 좌표계에 나타낼 수 없는 수치(복소수 $z$에 대하여 $z=Im(x)$로 나타내기도 함. $i=\sqrt{-1}$

\item \textbf{real number/실수}:
직선으로 된 좌표계에 표시 가능한 수치(복소수 $z$에 대하여 $z=Re(x)$로 나타내기도 한다)

\item \textbf{complex number/복소수}:
허수와 실수의 합. 주의할 점은 실수와 허수는 곱하기는 가능하지만 더하기가 불가능하다는 점. 따라서 $i3$와 같은 수는 pure imaginary number이고, 일반적으로 복소수라 함은 $3+i4$ 형식이다. 자세한 내용은 복소수의 정의 참조.

\item \textbf{irrational number/무리수}:
유리수가 아닌 실수. $\sqrt{2}$같은 수로써, 좌표계에 나타낼 수는 있지만, 유리수는 아니다.

\item \textbf{rational number/유리수}:
두 정수의 몫으로 나타낼 수 있는 수(fraction을 포함한 수)

\item \textbf{integer/정수}:
분모가 1인 자연수. 음의 정수와 양의 정수를 포함한다.

\item \textbf{natural number/자연수}:
오로지 양의 정수만 포함하고 있는 '숫자 세기'가 가능한 수.
\end{itemize}

\section{기호 및 등식}
\begin{itemize}
\item $!a$ = \textbf{factorial/계승} $a$부터 $1$까지 1씩 감소하여 곱하기를 나타냄.(작거나 같은 모든 양의 정수의 곱)
\item $\begin{pmatrix} a \\ b \end{pmatrix}$ = combinatorics. 표현할 때에는 `a choose b` 형식으로 표현함. 풀이법은 아래와 같음.
$$\begin{pmatrix} a\\b \end{pmatrix} = \dfrac{a!}{b!(a-b)!}$$
\item $\Sigma$ = \textbf{Sum 혹은 시그마} $a$가 아래첨자 $k$부터 위첨자$n$까지 증가하며 매 증가마다 그 옆에 들어가있는 수식을 계산하고 합산함.
$$ \sum_{a=k}^{n} (a+b) $$
\end{itemize}



\section{기초}
\subsection{허수의 정의/Definition of Imaginary Number}
허수를 $i$라고 했을 때, 다음과 같은 규칙이 성립한다.
$$i^{2}=-1$$
$$i=\sqrt{-1}$$
그러나 허수도 일반적인 실수와 별반 다르지 않은 속성 또한 있다.
$$i^{0}=1, i^{1}=1$$
이를 허수의 본래 성질과 연관시켜 계산해보면...
$$i^{3}=i^{1}\cdot i^{2}=-1\cdot i$$

$$ \begin{tabular}{| l | l | l | l |}
\hline
$i^{1}$ & $i^{2}$ & $i^{3}$ & $i^{4}$ \\
\hline
i & -1 & -i & 1 \\
\hline
\end{tabular}
$$
이러한 표를 만들 수 있다. 따라서 이러한 표의 내용이 순환된다고 볼 수 있는데, 4제곱마다 1로 돌아오므로 일단 허수의 제곱을 계산하고자 할 때에는 일단 승을 4로 나눈 뒤에 그 나머지를 1~4 사이에서 찾아 계산하면 된다.\\
또한 $i^{4}\cdot i^{-1}=i^{3}$이므로 여기서 추측하여 $i^{-1}=-i$로 거꾸로 된 순서도 유도가 가능하다.
\subsection{복소수의 정의/Definition of Complex Number}
허수와 실수는 곱할 수 있다. 그래서 $i3, i\cdot \sqrt{2}$등의 값을 순수한 허수(pure imaginary number)라고 한다. 하지만 빼고 더할 수는 없다. 그래서 $3+2i$같은 값은 순수한 허수가 아니다. 대신 새로운 체계를 만들어서 이를 표시하는데, 허수와 실수가 더해진 값을 복소수라고 한다.
$$a,b{\textsf{를 자연수(실수/real number)라고 할 때,}}$$
$$a\cdot i = {\textsf{pure imaginary number}}$$
$$a+b\cdot i = {\textsf{not a pure imaginary number, complex number}}$$
복소수 또는 허수는 실제로 계산할 수 없는 값이므로, 실수로 된 좌표계에 나타낼 수 없다. 다만 허수와 실수의 좌표계를 함께 가진 복합 좌표계(예를 들면 실수를 가로축으로 놓고, 허수를 세로축으로 놓는 등의)위에는 표현이 가능하다. 앞에서 배운 표기법을 활용하면 아래와 같이 표기 가능하다.
$$z=3+2i = Re(3), Im(2)$$
따라서 가로축을 실수로 잡고 세로축을 허수로 잡은 좌표계 위에서 (3,2)의 좌표를 가진다. \\ 더하기는 안되지만 곱하기는 가능하기 때문에, 수의 분배도 역시 정상적으로 이루어짐에 주의하자.
$$-4(13+5i)=-4(13)+(-4)(5i)=-52-20i$$

\subsection{Long Division of Polynomials 다항식의 나눗셈(전통적인 방법)}
나누고자 하는 대상 다항식의 가장 차수(degree)가 높은 부분부터 나눗셈한다.
\subsubsection{예제1}
$$\frac{x^{2}+3x+6}{x+1}$$
$$\Downarrow$$
$$\polylongdiv{x^{2}+3x+6}{x+1}$$
\begin{enumerate}
\item $x^{2}$가 가장 높은 차수이므로 x를 곱하여 차수를 맞추어주고 $(x+1)x = x^{2}+x$로 가장 먼저 나눈다: 몫에 x가 더해진다.
\item $(x^{2}+3x+6)-(x^{2}+x)=2x+6$이 된다.
\item 다음 가장 높은 차수는 $x^{1}$이므로 2를 곱하여 $(x+1)2 = 2x+2$로 나눈다: 몫에 2가 더해진다.
\item $(2x+6)-(2x+2) = 4$이므로 몫은 $x+2$이고, 나머지는 $4$이다.
\end{enumerate}
\begin{quote}
주의점: 예제는 덧셈이지만 가장 차수가 높은 항에서만 어긋나지 않는다면 뺄셈도 가능하다. 다음의 예제를 참조하자
\end{quote}

\subsubsection{예제2(음수가 있을 때)}
$$\frac{x^{2}+x+1}{x+2}$$
$$\Downarrow$$
$$\polylongdiv{x^{2}+x+1}{x+2}$$
\begin{enumerate}
\item $x^{2}$가 가장 높은 차수이므로 $x$를 곱하여 차수를 맞춰준다. $(x+2)x=x^{2}+2x$
\item $(x^{2}+x+1)-(x^{2}+2x)=-x+1$이 된다: 몫에 $x$가 더해진다.
\item 음수(-)가 나왔지만 당황하지 말고 마이너스로 곱한다. 정수부에는 문제가 없으므로 음수로 된 정수만 곱해주자. $-1(x+2)=-x-2$가 된다: 몫에 $-1$이 더해진다.
\item 앞에 남아있던 수($-x+1$)에서 빼주면 된다.\textbf{왜냐면 나눗셈은 기본적으로 마이너스 계산이기 때문이다.} $(-x+1)-(-x-2)=3$이므로 몫은 $x-1$이고 나머지는 $3$이다.
\end{enumerate}
\begin{quote}
{\color{red} 주의점}: long division을 하게되다보면 자연스럽게 음수와 양수가 뒤섞이게 되는데, 앞선 결과값의 음수를 가끔씩 양수로 잘못 표기하여 이어 계산하는 경우가 있다. 이것 때문에 문제를 수십번씩 다시풀었다. 바보가 되기 싫다면 다음 차수로 넘어가기 전에 꼭 음수와 양수가 맞는지 두번 확인하자.\\
{\color{red} 주의점}: 당연한 얘기지만, 각종 차수와 정수부, 미지수를 표기하다보면 사람이라 실수가 생길수밖에 없다. 항상 문제를 풀기 전에 옮겨 적은 문제에 오기는 없는지 꼭 두번 이상 확인해야된다. 지금이야 별 문제 없지만, 나중에 실전에서 수학으로 문제 해결시 정확도가 매우 중요해질 것을 생각하면, 지금부터 버릇을 들여야 할 것 같다.
\end{quote}

\subsection{synthetic division of polynomials 다항식의 알고리즘적 나눗셈}
방법이 조금 복잡하므로 차라리 유튜브 영상을 볼것
\vspace{12pt}\\
\url{https://www.khanacademy.org/math/algebra2/arithmetic-with-polynomials/modal/v/synthetic-division}
\vspace{12pt}\\
\url{https://www.khanacademy.org/math/algebra2/arithmetic-with-polynomials/modal/v/synthetic-division-example-2}\\
Salman Khan도 그다지 즐겨쓰는 방법은 아니라고 했으니 크게 신경쓰지 않아도 될듯. 어차피 $x\pm a$형태의 이항식으로 나눌 때에만 사용 가능하다고 하니 정 필요할 때에만 사용하자.

\subsection{Polynomial Remainder Theorem}
$$\frac{f(x)}{x-a}={\textsf{remainder of}} f(a)$$
단순히 표현식만으로는 이해가 잘 안될수도 있는데 예를 들어 $f(x) = 3x^{2}-4x+7$이라고 한다면, 이를 $x-1$로 나눌 경우 그 나머지는 $f(1)$과 같다는 뜻이다.
$$\polylongdiv{3x^{2}-4x+7}{x-1}$$
위의 결과를 차용하면, 결국 $\frac{f(x)}{x-1}=...6$이므로 $6=f(1)$인 것이다. \textbf{즉 다항식 나눗셈의 나머지로 다항식의 값을 알아낼 수 있다는 것이다!} 혹은 그 반대로 값을 대입하기만 해도 다항식의 나머지를 알아낼 수 있다.(물론 몫은 알 수 없다) 그리고 나머지를 알 수 있기 때문에, $f(x)$를 가지고 특정한 이항식$(x-a)$이 적정한 해(factor)인지 아닌지를 알 수 있다. 대입 후 0이 나오면(나머지가 없으면) 적정한 해가 맞는 것이고, 나머지가 있으면 해가 아니다.

\subsection{Arithmetic of Series 수열의 산수}
\subsubsection{수열의 덧셈}
두 수열의 합은 아래와 같이 정리 가능하다.
$$S_n=\{1,2,3,\dots,n\}$$
$$s_n=1+2+3+\dots+n$$
$$s_n=n+(n-1)+(n-2)+(n-3)+\dots+1$$
따라서,
$$2S_n=(n+1)+(2+n-1)+(3+n-2)+\dots$$
$$=(n+1)+(n+1)+(n+1)+\dots$$
그렇기 때문에 결국 1개 수열의 총합을 구하는 식은 아래와 같다
$$\frac{2S_n}{2}=\frac{n(n+1)}{2}$$
여기서 다소 흥미로운 결론이 도출되는데, 위의 식을 정리하면 $n\cdot\frac{n+1}{2}$로도 볼 수 있고, 여기서 분자인 $n+1$은 결국 수열의 첫 값과 끝 값을 더한 것이다. 이를 $2$로 나눈 것은 또한 두 값에 대한 중간 값을 구한 것으로 볼 수 있다. 따라서 어떤 수열의 합은 해당 수열의 처음과 끝을 더한 후 중간값을 구하여 해당 수열의 크기 만큼 곱한 것이다. 이는 $1$로 시작하고 증가하는 값이 $1$인 수열에 해당하는 것이므로, 위의 수식에는 일부 드러나지 않는 값이 있다. 이를 모든 경우에 사용할 수 있도록 일반화 하면 아래와 같다.
$$S_n=\frac{n(2\cdot\textsf{initial value}+(n-1)\cdot\textsf{step})}{2}$$ step이 들어가서 조금 달라보이긴 하지만, 요점은 결국 첫 값과 끝 값을 더한 뒤에 평균 값을 구하여 수열의 크기만큼 곱한 것이다. step을 더한 것은 어디까지나 마지막 값을 구하기 위한 것이고 initial value가 2개인 것은 마지막 값에도 initial value가 포함되기 때문이다. 결국 수열의 합은 아래와 같이 더욱 쉽게 표현할 수 있다.
$$\Downarrow$$
$$S_n=n\cdot\frac{\textsf{initial value} + \textsf{end value}}{2}$$
요지는 주어진 step값 또는 function을 통해 제대로 된 초기값과 마지막 값을 도출하는 것이다. 그렇게 하면 수열의 합은 쉽게 구할 수 있다.
{\color{red}씨발 음수} 복잡한 산수를 하거나 계산기를 사용할 때 숫자입력 틀리지 않도록 주의...$\Sigma$의 수식을 주지 않고 수열의 예제만 준 뒤에 수식을 유도하여 덧셈을 계산하도록 하는 경우가 있는데, 이 경우에는 수식을 항상 만들어보고 계산해야된다. 안그러면 수열의 마지막 수를 계산할 때 틀리는 경우가 종종 있다.\\
문제 풀어보기 \url{https://www.khanacademy.org/math/algebra2/sequences-and-series/modal/e/arithmetic_series}\\

\end{document}
