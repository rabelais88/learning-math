\documentclass{article}
\usepackage{kotex}
\usepackage{color}
\usepackage{mathtools}
\usepackage{hyperref}
\usepackage[margin=0.5in]{geometry}
\usepackage{polynom}
\title{Algebra II \\대수학2}
\author{박성렬}

\begin{document}
\maketitle

\section {용어}
\begin {itemize}
\item ?mial: ?항식
\begin {tabular} {| l | l | l | l |}
\hline
monomial & 1항식 & $a$\\
\hline
binomial & 2항식 & $a+b$\\
\hline
trinomial & 3항식 & $a+b+c$\\
\hline
polynomial & 다항식 & $a+...$\\
\hline
\end {tabular}

\item binomial theorem - 2항식의 법칙 -
$$(a+b)^{n}= \sum_{k=0}^{n} \begin{pmatrix}n \\ k \end{pmatrix} a^{n-k}b^{k}$$
$$\begin{pmatrix} n \\k \end{pmatrix}=\dfrac{n!}{k!(n-k)!} $$
\end {itemize}

\section {기호 및 등식}
\begin {itemize}
\item $!a$ = factorial. 계승. $a$부터 $1$까지 1씩 감소하여 곱하기를 나타냄.(작거나 같은 모든 양의 정수의 곱)
\item $\begin{pmatrix} a \\ b \end{pmatrix}$ = combinatorics. 표현할 때에는 `a choose b` 형식으로 표현함. 풀이법은 아래와 같음.
$$\begin{pmatrix} a\\b \end{pmatrix} = \dfrac{a!}{b!(a-b)!}$$
\item $\Sigma$ = Sum. 혹은 시그마. $a$가 아래첨자 $k$부터 위첨자$n$까지 증가하며 매 증가마다 그 옆에 들어가있는 수식을 계산하고 합산함.
$$ \sum_{a=k}^{n} (a+b) $$
\end {itemize}



\section {기초}

\subsection {long division of polynomials 다항식의 나눗셈(전통적인 방법)}
나누고자 하는 대상 다항식의 가장 차수(degree)가 높은 부분부터 나눗셈한다.
\subsubsection{예제1}
$$\frac{x^{2}+3x+6}{x+1}$$
$$\polylongdiv{x^{2}+3x+6}{x+1}$$
\begin {enumerate}
\item $x^{2}$가 가장 높은 차수이므로 x를 곱하여 차수를 맞추어주고 $(x+1)x = x^{2}+x$로 가장 먼저 나눈다: 몫에 x가 더해진다.
\item $(x^{2}+3x+6)-(x^{2}+x)=2x+6$이 된다.
\item 다음 가장 높은 차수는 $x^{1}$이므로 2를 곱하여 $(x+1)2 = 2x+2$로 나눈다: 몫에 2가 더해진다.
\item $(2x+6)-(2x+2) = 4$이므로 몫은 $x+2$이고, 나머지는 $4$이다.
\end {enumerate}
\begin{quote}
주의점: 예제는 덧셈이지만 가장 차수가 높은 항에서만 어긋나지 않는다면 뺄셈도 가능하다. 다음의 예제를 참조하자
\end{quote}

\subsubsection{예제2(음수가 있을 때)}
$$\frac{x^{2}+x+1}{x+2}$$
$$\polylongdiv{x^{2}+x+1}{x+2}$$
\begin{enumerate}
\item $x^{2}$가 가장 높은 차수이므로 $x$를 곱하여 차수를 맞춰준다. $(x+2)x=x^{2}+2x$
\item $(x^{2}+x+1)-(x^{2}+2x)=-x+1$이 된다: 몫에 $x$가 더해진다.
\item 음수(-)가 나왔지만 당황하지 말고 마이너스로 곱한다. 이미 실수부에는 문제가 없으므로 음수로 된 정수만 곱해주자. $-1(x+2)=-x-2$가 된다: 몫에 $-1$이 더해진다.
\item 앞에 남아있던 수($-x+1$)에서 빼주면 된다.\textbf{왜냐면 나눗셈은 기본적으로 마이너스 계산이기 때문이다.} $(-x+1)-(-x-2)=3$이므로 몫은 $x-1$이고 나머지는 $3$이다.
\end{enumerate}
\begin{quote}
{\color{red} 주의점}: long division을 하게되다보면 자연스럽게 음수와 양수가 뒤섞이게 되는데, 앞선 결과값의 음수를 가끔씩 양수로 잘못 표기하여 이어 계산하는 경우가 있다. 이것 때문에 문제를 수십번씩 다시풀었다. 바보가 되기 싫다면 다음 차수로 넘어가기 전에 꼭 음수와 양수가 맞는지 두번 확인하자.\\
{\color{red} 주의점}: 당연한 얘기지만, 각종 차수와 정수부, 미지수를 표기하다보면 사람이라 실수가 생길수밖에 없다. 항상 문제를 풀기 전에 옮겨 적은 문제에 오기는 없는지 꼭 두번 이상 확인해야된다. 지금이야 별 문제 없지만, 나중에 실전에서 수학으로 문제 해결시 정확도가 매우 중요해질 것을 생각하면, 지금부터 버릇을 들여야 할 것 같다.
\end{quote}

\subsection {synthetic division of polynomials 다항식의 알고리즘적 나눗셈}
방법이 조금 복잡하므로 차라리 유튜브 영상을 볼것
\vspace{12pt}\\
\url{https://www.khanacademy.org/math/algebra2/arithmetic-with-polynomials/modal/v/synthetic-division}
\vspace{12pt}\\
\url{https://www.khanacademy.org/math/algebra2/arithmetic-with-polynomials/modal/v/synthetic-division-example-2}\\
Salman Khan도 그다지 즐겨쓰는 방법은 아니라고 했으니 크게 신경쓰지 않아도 될듯. $x+-a$형태의 

\subsection{Polynomial Remainder Theorem}
$$\frac{f(x)}{x-a}={\textsf{remainder of}} f(a)$$
단순히 표현식만으로는 이해가 잘 안될수도 있는데 예를 들어 $f(x) = 3x^{2}-4x+7$이라고 한다면, 이를 $x-1$로 나눌 경우 그 나머지는 $f(1)$과 같다는 뜻이다.
$$\polylongdiv{3x^{2}-4x+7}{x-1}$$
위의 결과를 차용하면, 결국 $\frac{f(x)}{x-1}=...6$이므로 $6=f(1)$인 것이다. \textbf{즉 다항식 나눗셈의 나머지로 다항식의 값을 알아낼 수 있다는 것이다!} 혹은 그 반대로 값을 대입하기만 해도 다항식의 나머지를 알아낼 수 있다.(물론 몫은 알 수 없다)

\end{document}
