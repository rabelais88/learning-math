\documentclass{article}
\usepackage{kotex} % for korean support
\usepackage{color}
\usepackage{mathtools} % for pmatrix command - 행렬 표시
\usepackage{hyperref} % for hyperlink
\usepackage[margin=0.5in]{geometry} % for document margin
\usepackage{polynom} % automatic polynomial calculations(divisions, more specifically)
\usepackage{tikz}
\title{Highschool Geometry 고등수학 도형}
\author{박성렬}

\begin{document}
\maketitle
\section{기본개념}
\subsection{회전 Rotation}
도형의 회전을 알아보고자 할 때에는, 도형의 회전이 일어났다고 예상되는 지점에 점을 먼저 찍어야 한다. 회전이 이루어난 점을 기준으로 기존의 점 A와 새로운 점 'A를 찍어서 연결하면 회전이 일어난 각이 나온다.\\
$$
\begin{tikzpicture}

\draw (1,3) -- (-1,1.5) -- (2,1.5);
\node at (1,3.5) {\\A};
\node at (-1,1.5) {};
\node at (2.5,1.5) {\\`A};
\draw (-0.234,2.1428) arc (40.0022:0:1);
\end{tikzpicture}
$$
회전을 수로 나타날 때 직관과는 반대로 양수가 시계 반대방향이고 음수가 시계 방향이므로 유의하자.

\end{document}